\documentclass[a4paper,12pt,fleqn,oneside]{article}
\usepackage{graphicx}
\usepackage{etex}
\usepackage[latin1]{inputenc}
\usepackage[ngerman]{babel}
\usepackage{ae,aecompl}
\usepackage[T1]{fontenc}
\usepackage{ngerman}
%\usepackage{fleqn}
\usepackage{ulem}
\usepackage{amssymb}
\usepackage[locale=DE, per-mode=fraction, quotient-mode=fraction, group-minimum-digits=6]{siunitx}
\usepackage{tabularx}
\usepackage{bm}
\usepackage{booktabs}
\usepackage{color}
\usepackage{pictex}
\usepackage[left=2.5cm,right=2.5cm,top=2cm,bottom=2cm,includeheadfoot]{geometry}
\usepackage[section]{placeins}
\usepackage{xspace}
\usepackage{multirow}
\usepackage{lastpage}
\usepackage{fancyhdr}
\usepackage{graphicx}
\usepackage{esvect}
\usepackage{pgfplots}
\usepackage[ngerman]{babel}
\usepackage {graphics}
\usepackage {graphicx}
\usepackage{tikz}
\usepackage{amsmath}
\usepackage{autonum}


\setlength{\headheight}{15pt}
\pagestyle{fancy}
\fancyfoot[C]{Seite \thepage{} von \pageref{LastPage}}
\linespread{1.5}
\author{Dominik Eisele}
\title{Laborbericht}
\date{\today}


%\begingroup\makeatletter\ifx\SetFigFontNFSS\undefined%
%\gdef\SetFigFontNFSS#1#2#3#4#5{%
%  \reset@font\fontsize{#1}{#2pt}%
%  \fontfamily{#3}\fontseries{#4}\fontshape{#5}%
%  \selectfont}%
%\fi\endgroup%


% Zus�tzliche Spalten mit variabler Breite fur tabularx
%\newcolumntype{L}{>{\raggedright\arraybackslash}X} % linksb�ndig
%\newcolumntype{C}{>{\centering\arraybackslash}X} % zentriert
%\newcolumntype{R}{>{\raggedleft\arraybackslash}X} % rechtsb�ndig


\newcolumntype{L}[1]{>{\raggedright\arraybackslash}p{#1}} % linksb�ndig mit Breitenangabe
\newcolumntype{C}[1]{>{\centering\arraybackslash}p{#1}} % zentriert mit Breitenangabe
\newcolumntype{R}[1]{>{\raggedleft\arraybackslash}p{#1}} % rechtsb�ndig mit Breitenangabe


\setlength{\tabcolsep}{0pt}
\renewcommand{\arraystretch}{1}


%\renewcommand*\contentsname{Gliederung}


\let\oldsqrt\sqrt
\def\sqrt{\mathpalette\DHLhksqrt}
\def\DHLhksqrt#1#2{\setbox0=\hbox{$#1\oldsqrt{#2\,}$}\dimen0=\ht0
\advance\dimen0-0.3\ht0
%0.3 ist das Ma� f�r die Hakenl�nge, relativ zum Inhalt der Wurzel
\setbox2=\hbox{\vrule height\ht0 depth -\dimen0}%
{\box0\lower0.4pt\box2}}



\begin{document}

\begin{titlepage}
	\begin{flushleft}
		\vspace*{2\baselineskip}
		{\fontsize{16}{19.2}\selectfont Laborbericht Physik TGE12/2 A}\\[4\baselineskip]
		\begin{tabularx}{\textwidth}{rp{5px}X}
			{\fontsize{16}{19.2}\selectfont Titel:}&&{\fontsize{16}{19.2}\selectfont Millikan Versuch}
		\end{tabularx}
		\\[5\baselineskip]
		\setlength{\tabcolsep}{0pt}
		\renewcommand{\arraystretch}{1,25}
		\begin{tabular}{lp{5px}l}
			{\fontsize{14}{16.8}\selectfont Bearbeiter:}&&{\fontsize{14}{16.8}\selectfont Dominik Eisele{}}\\
			{\fontsize{14}{16.8}\selectfont Mitarbeiter:}&&{\fontsize{14}{16.8}\selectfont}\\
			{\fontsize{14}{16.8}\selectfont Datum Versuchsdurchf�hrung:}&&{\fontsize{14}{16.8}\selectfont 11.04.2016}\\
			{\fontsize{14}{16.8}\selectfont Datum Abgabe:}&&{\fontsize{14}{16.8}\selectfont 20.06.2016}
		\end{tabular}
		\\[2\baselineskip]
		{\fontsize{14}{16.8}\selectfont Ich erkl�re an Eides statt, den vorliegenden Laborbericht selbst angefertigt zu haben. Alle fremden Quellen wurden in diesem Laborbericht benannt.}
		\\[2\baselineskip]
		{\fontsize{14}{16.8}\selectfont Hochdorf, \today $  $ Dominik Eisele}
	\end{flushleft}
\end{titlepage}

\setlength{\tabcolsep}{7pt}
\renewcommand{\arraystretch}{1,7}

\newpage
\tableofcontents
\newpage


\section{Einf�hrung}
	Bei dem Versuch "`Elektrolytischer Trog"' ergibt sich der Verlauf der Feldlinien zwischen zwei Elektroden. Auf diesen Verlauf
	kommt man, indem man zwischen einem Plattenkondensator, der sich in einem Elektrolyt befindet, �quipotentiallinien(-fl�chen)
	sucht und sich diese Verl�ufe notiert. Durch diese �quipotentiallinien muss man jeweils eine Orthogonale einzeichnen, und
	man erh�lt den Verlauf von Feldlinien in einem Plattenkondensator.\\
	An den Elektroden wurde eine Wechselspannung  angelegt, da sich so keine Debye-Schicht (Elektrochemische 
	Raumladungsdoppelschicht), wie bei einer Gleichspannung, ausbreiten konnte. Bei dieser Raumladungsdoppelschicht 
	w�rde zus�tzlich ca. \SI{1}{\volt} abfallen.

\subsection{Formeln}
	Betrag der elektrischen Feldst�rke:
	\[E=\frac{\text{Potentialdifferenz }\Delta\varphi}{\text{Abstand der Potentiallinien }\Delta s}\]

\newpage
\section{Material und Methoden}

\subsection{Material}
	F�r den Versuch verwendete Materialien:
	\begin{itemize}
		\item 1 $\times$ Netzteil
		\item 3 $\times$ Kabel
		\item 1 $\times$ Spannungsmesser
		\item 1 $\times$ Messspitze zur Spannungsmessung
		\item 1 $\times$ Elektrolytischer Trog mit planen Elektroden
		\item 1 $\times$ Metallring
		\item Millimeterpapier
	\end{itemize}

\subsection{Aufbau}
	Ein mit Wasser gef�llter elektrolytischer Trog wie er in Abbildung \ref{fig:skizze_elektrolytischer_trog} zu sehen ist, besteht aus
	einer Kunststoffwanne, die an beiden Seiten jeweils eine plan liegende Elektrode besitzt. Diese Elektroden sind so platziert, dass
	sie in das, als Elektrolyt verwendete, Leitungswasser hineinragen. An diese  Elektroden wurde eine Wechselspannung angelegt.
	Au�erdem wurde noch ein Multimeter installiert, sodass man mit einer Messspitze die Potentialdifferenz zwischen einer Elektrode
	und einer beliebigen Stelle im Elektrolyt messen konnte.\\
	In der zweiten Messreihe wurde noch ein Metallring in den elektrolytischen Trog gelegt, sodass das elektrische Feld durch ihn gest�rt
	wird.
	

	
%	\begin{figure}[H!]
%		\centering
%		\scalebox{.9}{\input{skizze_elektrolytischer_trog}}
%		\caption{Skizze des Versuchsaufbaus}
%		\label{fig:skizze_elektrolytischer_trog}
%	\end{figure}

\newpage
\subsection{Durchf�hrung}
	Nachdem der Versuch aufgebaut wurde, wurden mit der Messspitze �quipotentiallinien im elektrischen Feld gesucht. Diese 
	�quipotentiallinien wurden auf Millimeterpapier �bertragen, die Spannung betrug hier dabei \SI{2}{\volt}, \SI{3}{\volt},
	\SI{4}{\volt}, \SI{5}{\volt}, \SI{6}{\volt}, \SI{7}{\volt} und \SI{8}{\volt}.\\
	Anschlie�end wurden die Magnetfeldlinien eingezeichnet und es wurde die Feldst�rke $E$ f�r drei Punkte berechnet.
	Der Versuch wurde daraufhin mit einem, sich im elektrolytischem Trog befindenden, Metallring wiederholt.


\newpage
\section{Messwerte}
	In Abbildung \ref{fig:Diagramm1} sind die Messpunkte und die �quipotentiallinien f�r \SI{2}{\volt} (schwarz), \SI{3}{\volt} (pink),
	\SI{4}{\volt} (blau), \SI{5}{\volt} (hellgr�n), \SI{6}{\volt} (rot), \SI{7}{\volt} (lila) und \SI{8}{\volt} (orange) eingezeichnet.
	Dazu wurden au�erdem die magnetischen Feldlinien (dunkelgr�n) eingezeichnet, die zwischen den beiden Kondensatorplatten,
	beim Anlegen einer Spannung, entstehen, und jeweils noch drei Punkte zur Berechnung der elektrischen Feldst�rke $E$.\\
	In Abbildung \ref{fig:Diagramm2} sind ebenfalls die Messpunkte, �quipotentiallinien und Feldlinien eingezeichnet, die Farben entsprechen
	den aus Abbildung \ref{fig:Diagramm1}. Hinzu kommt in Abbildung \ref{fig:Diagramm2} der Metallring, der im Inneren des Rings
	eine �quipotentialfl�che besa�, deren Spannung \SI{4.3}{\volt} betrug. Der Metallring selbst ist ebenfalls eine �quipotentiallinie,
	dessen Spannung \SI{4.7}{\volt} betrug.
			

	
%	\begin{figure}
%		\input{messwerte1}
%	\end{figure}
%
%	\begin{figure}
%		\input{messwerte2}
%	\end{figure}

	

\newpage
\section{Auswertung}
	\subsection{Betrag der elektrischen Feldst�rke $E$}
	Die ben�tigte Formel zu Berechnung der elektrischen Feldst�rke $E$ lautet:
	\[E=\frac{\text{Potentialdifferenz }\Delta\varphi}{\text{Abstand der Potentiallinien }\Delta s}\] 
	F�r den Abstand der Potentiallinie  wurde die L�nge Feldlinien gemessen. 
	
	\subsubsection{Messreihe 1}
	Punkt $P_1 \left(\num{1.8} \mid 2\right)$,  Punkt $P_2 \left(\num{5,5} \mid \num{4,4}\right)$, Punkt $P_3 \left(\num{9.6} \mid 8\right)$:
		\[E_{P_1} = \frac{\SI{10}{\volt}}{\SI{0,11}{\metre}} = \SI{90,9091}{\volt \per \meter}\] 
		\[E_{P_2} = \frac{\SI{10}{\volt}}{\SI{0.115}{\metre}} = \SI{86,9565}{\volt \per \meter}\]
		\[E_{P_3} = \frac{\SI{10}{\volt}}{\SI{0.12}{\metre}} = \SI{83,3333}{\volt \per \meter}\]
	

	\subsubsection{Messreihe 2}
	Punkt $P_1 \left(\num{1,3} \mid \num{5,5}\right)$,  Punkt $P_2 \left(\num{6.2} \mid \num{1,35}\right)$, Punkt $P_3 \left(\num{7.38} \mid \num{-2,05}\right)$:
		\[E_{P_1} = \frac{\SI{10}{\volt}}{\SI{0,12}{\metre}} = \SI{83,3333}{\volt \per \meter}\] 
		\[E_{P_2} = \frac{\SI{10}{\volt}}{\SI{0.11}{\metre}} = \SI{90,9091}{\volt \per \meter}\]
		\[E_{P_3} = \frac{\SI{10}{\volt}}{\SI{0.11}{\metre}} = \SI{90,9091}{\volt \per \meter}\]

	\subsection{Beobachtung zum Metallring}
	Der in Messreihe 2 hinzugef�gte Metallring besa� ein einheitliches Potential von  \SI{4,3}{\volt}. Diese einheitliches Potential
	entsteht, da der Ring als Leiter keine Potentialdifferenz besitzt. Dadurch ist die Ringinnenfl�che komplett von den beiden Elektroden abgeschirmt,
	und es wirkt nur der Ring als �quipotentiallinie auf das Ringinnere. Dadurch entsteht keine Potentialdifferenz im Ringinneren.

\end{document}